
\chapter{Il problema affrontato}

\medskip

L'obiettivo del presente elaborato è mostrare il funzionamento di un blog decentralizzato sviluppato tramite tecnologia {\tt Solid}, mettendo in evidenza i benefici che un'applicazione {\tt Solid} può apportare all'utente.

\medskip

Il progetto è composto da due applicazioni differenti e svincolate tra di loro. La prima è chiamata {\tt my-solid-blog}; tale applicazione offre due differenti tipologie di servizi: la prima consiste nel permettere la creazione di un proprio blog, salvando tutte le informazioni di interesse all'interno del suo {\tt Pod}. In questo caso, per poter ottenere il diritto in scrittura sui dati presenti all'interno del {\tt Pod}, l'applicazione richiede preventivamente all'utente di autenticarsi con il proprio {\tt Solid Idenitity Provider}; il secondo servizio implementato in {\tt my-solid-blog} consiste nel permettere la lettura di blog di altre persone una volta conosciuta la loro {\tt webId}. Per questa seconda funzionalità la procedura di autenticazione  con il proprio {\tt Solid Identity Provider} non è richiesta.

\medskip

La seconda applicazione trattata in questo elaborato di tesi, chiamata {\tt blog-validator}, permette, su richiesta dell'utente, di effettuare controlli riguardo l'autenticità dei dati mostrati da {\tt my-solid-blog}. Tale applicazione ha lo scopo di evidenziare la semplicità con cui un'operazione di questo tipo può essere implementata in {\tt Solid}.

\bigskip

\section{Obiettivi del progetto}

\medskip

Il sistema {\tt SADeB} consiste di due applicazioni differenti: {\tt my-solid-blog} e {\tt blog-validator}. Vengono ora elencate le specifiche relative all'applicazione\\{\tt my-solid-blog}.

\medskip

\begin{itemize}
	\item L'applicazione {\tt my-solid-blog} deve essere completamente decentralizzata, rispettando i principi della tecnologia {\tt Solid}.
	\item Una volta avviata l'applicazione, l'utente deve poter scegliere la modalità con cui interagire con l'app, indicando se vuole autenticarsi con il proprio {\tt Solid Identity Provider} e modificare, creare ed eliminare i dati presenti all'interno del proprio blog, oppure se inserire la {\tt webId} di un altro utente per visualizzare il blog di quest'ultimo.
	\item L'applicazione deve implementare un meccanismo di autenticazione con il server {\tt Solid}, sfruttando le librerie messe a disposizione da {\tt Inrupt}.
	\item Una volta che l'utente si è autenticato con il proprio {\tt Solid Identity Provider}, l'applicazione {\tt my-solid-blog} deve poter controllare se esiste all'interno del {\tt Pod} dell'utente un {\tt SolidDataset} relativo al blog. In questo caso, {\tt my-solid-blog} deve caricare tale {\tt SolidDataset} per poter leggere e modificare i dati contenuti in esso. Altrimenti l'applicazione deve creare tale dataset all'interno del {\tt Pod} dell'utente. Il nome del {\tt SolidDataset} utilizzato dall'applicazione è {\tt articlelist.ttl}.
	\item I dati contenuti all'interno del {\tt SolidDataset} {\tt articlelist.ttl} utilizzato da\\ {\tt my-solid-blog}, devono rappresentare degli articoli da pubblicare sul blog\\relativo all'utente che è proprietario del {\tt Pod}. Per poter rappresentare tali dati, l'applicazione deve far uso del vocabolario {\tt RDF} {\tt Schema}.
	\item Ogni articolo deve avere un titolo, un testo che ne rappresenti il contenuto, una data di creazione e un identificatore univoco. Ogni articolo è un oggetto e un'istanza della classe {\tt TextDigitalDocument} del vocabolario {\tt Schema}. L'applicazione deve permettere all'utente di visualizzare il titolo, il contenuto e la data di creazione di ciascun articolo presente.
	\item L'applicazione {\tt my-solid-blog} deve permettere all'utente autenticato di leggere, modificare e cancellare i dati, cioé gli articoli, presenti all'interno del {\tt SolidDataset} {\tt Articlelist.ttl}. A tal proposito {\tt my-solid-blog} deve utilizzare le librerie {\tt Inrupt} per poter svolgere queste operazioni sui dati contenuti all'interno del {\tt Pod}. 
	\item L'utente autenticato deve essere, inoltre, in grado di scrivere nuovi articoli all'interno del proprio blog.
	\item Un utente può scegliere di accedere al contenuto di un blog di un secondo utente, inserendo la sua {\tt webId} senza passare per alcun processo di autenticazione. In questo caso, se esiste il {\tt SolidDataset} {\tt articlelist.ttl} all'interno del {\tt Pod} corrispondente alla {\tt webId} inserita, l'applicazione deve permettere all'utente di visualizzare gli articoli relativi alla {\tt webId} inserita.
	\item Se l'utente accede ad un blog inserendo la {\tt webId} corrispondente al {\tt Pod} ove questo è memorizzato, senza quindi passare per un processo di autenticazione, non deve essere in grado di modificare il contenuto degli articoli all'interno di tale {\tt Pod}. Inoltre l'utente non autenticato non deve avere i permessi necessari per scrivere nuovi articoli all'interno del {\tt SolidDataset} {\tt articlelist.ttl} corrispondente alla {\tt webId} inserita.
	\item L'applicazione {\tt my-solid-blog} deve poter comunicare tramite il protocollo {\tt HTTP} con il server {\tt blog-validator}, per potergli inviare, su richiesta dell'utente, i dati necessari ad effettuare un controllo all'interno del {\tt Pod} del proprietario di questi dati. Per implementare tale funzionalità, {\tt my-solid-blog} deve renderizzare per ogni articolo un pulsante {\tt Check} per innescare tale controllo sull'autenticità dei contenuti visualizzati. Il controllo è riferito al singolo articolo selezionato, e non all'intero {\tt SolidDataset}.
	\item Ogni utente, indipendentemente dal fatto che abbia completato l'operazione di autenticazione o meno con il server {\tt Solid}, deve essere in grado di richiedere un controllo sull'autenticità dei contenuti mostrati da {\tt my-solid-blog}.
	\item L'applicazione {\tt my-solid-blog} deve mettere a disposizione un link che rimandi a {\tt blog-validator} per poter permettere a tutti gli utenti, autenticati e non, di verificare che eventuali controlli sull'autenticità dei dati effettuati siano stati eseguiti correttamente.
	\item Ogni utente deve poter visualizzare l'URL associato all'articolo che sta leggendo, ovvero l'URL ove tale risorsa è stata memorizzata.
	\item Le interfacce grafiche dell'applicazione, create tramite {\tt React}, devono essere semplici e intuitive, in modo tale da rendere l'applicazione user friendly.
\end{itemize}

\bigskip

Vengono qui di seguito elencate le specifiche relative all'applicazione {\tt blog-validator}.

\begin{itemize}
	\item Tramite il protocollo {\tt HTTP}, l'applicazione deve poter ricevere da\\{\tt my-solid-blog} le informazioni necessarie per effettuare un controllo sull'autenticità dei dati mostrati. In particolare {\tt blog-validator} si occupa di verificare che un singolo articolo, scelto dall'utente, sia effettivamente presente all'interno del {\tt Pod} dell'utente proprietario del blog. Una volta terminato tale controllo, l'applicazione {\tt blog-validator} deve inviare a {\tt my-solid-blog} l'esito del controllo stesso.
	\item I possibili esiti di un controllo sono i seguenti:
	
	\smallskip
	
	\textbf{1)} L'articolo in questione non esiste all'interno del {\tt Pod} del proprietario.\\
	\textbf{2)} L'articolo esiste, ma il contenuto risulta essere differente.\\
	\textbf{3)} L'articolo esiste e presenta lo stesso contenuto, ma la data in cui è stato pubblicato è differente.\\
	\textbf{4)} Il controllo sull'autenticità del contenuto ha dato un risultato positivo: in questo caso il server non ha rilevato alcuna anomalia.
	\item {\tt blog-validator} deve mettere a disposizione un URL per permettere all'utente di visualizzare i dati relativi ai controlli già effettuati sul {\tt Pod} di un utente. Questa funzionalità di {\tt blog-validator} serve ad assicurarsi che {\tt my-solid-blog} abbia inviato correttamente i dati relativi ai controlli effettuati, senza manometterli.
	\item {\tt blog-validator}, infine, deve poter permettere all'utente di effettuare controlli manualmente sulle risorse relative ai blog creati da {\tt my-solid-blog}; insererendo l'url del articolo che si vuole controllare e del {\tt SolidDataset} dove è memorizzato, {\tt blog-validator} deve mostrare all'utente contenuto, titolo e data di creazione dell'articolo, nel caso in cui questo sia effettivamente presente all'interno del {\tt Pod}.
\end{itemize}

\bigskip

\section{Modalità di accesso ai dati contenuti nel Pod}

\medskip

Nel corso dello svolgimento del progetto si è rivelata fondamentale la comprensione delle modalità con cui un'applicazione {\tt Solid} debba accedere ai dati contenuti all'interno del {\tt Pod}. Innanzitutto entrambe le applicazioni, {\tt my-solid-blog} e {\tt blog-validator}, utilizzano la libreria {\tt Inrupt} {\tt solid-client} per poter interagire con tali dati; la libreria mette a disposizione una serie di funzioni per poter scaricare {\tt SolidDataset} e per poter modificare gli oggetti al suo interno. 

\bigskip

Vengono qui mostrate le funzioni più importanti contenute in tale libreria:

\bigskip
\medskip

$getSolidDataset(url, options?): Promise<SolidDataset \& WithServerResourceInfo>$

\bigskip

Funzione che restituisce una {\tt promessa} che si risolve nel {\tt SolidDataset} corrispondente all'URL indicato; l'utente deve possedere il diritto in lettura su tale risorsa per poter caricare tale {\tt SolidDataset}. A tal scopo, se necessario, può passare alla funzione i dati relativi alla sessione di autenticazione con il {\tt Solid Identity Provider}, per poter acquisire tale diritto in lettura.

\bigskip
\medskip

$saveSolidDatasetAt<Dataset>(url, solidDataset, options?): Promise<Dataset\\ \& WithServerResourceInfo \& WithChangeLog>$

\bigskip

Tale funzione permette di salvare un determinato {\tt SolidDataset} all'interno dell'URL specificato; si possono passare a questa funzione anche i dati relativi alla sessione di autenticazione con il {\tt Solid Identity Provider} per acquisire diritto in scrittura sulle risorse del {\tt Pod}. Il possesso del diritto in scrittura è una condizione necessaria per poter utilizzare tale funzione. Restituisce una {\tt promessa} che si risolve nel nuovo {\tt SolidDataset} presente a seguito dell'esecuzione di questa funzione. Tale funzione, inoltre, può essere utilizzata per tenere traccia dei cambiamenti apportati all'interno di tale {\tt SolidDataset}.

\bigskip
\medskip

$createSolidDataset(): SolidDataset$

\bigskip

Permette di creare un nuovo {\tt SolidDataset}.

\bigskip
\medskip

$getThing(solidDataset, thingUrl, options?): ThingPersisted | null$

\bigskip

Ritorna un oggetto presente all'interno di un {\tt SolidDataset}, è necessario passare a tale funzione il {\tt SolidDataset} che contiene tale oggetto e l'URL relativo all'oggetto che si vuole ottenere. La funzione ritorna il valore {\tt null} se l'oggetto specificato non esiste.

\medskip
\bigskip

$getThingAll(solidDataset, options?): Thing[]$

\bigskip

Ritorna un array contenente tutti gli oggetti presenti all'interno di un {\tt SolidDataset}.

\bigskip
\medskip

$createThing(options): ThingPersisted$

\bigskip

Crea un nuovo oggetto.

\bigskip
\medskip

$setThing<Dataset>(solidDataset, thing): Dataset \& WithChangeLog$

\bigskip

Permette di aggiungere un nuovo oggetto all'interno di un {\tt SolidDataset}, rimpiazzando evenutali precedenti instanze di tale oggetto.

\bigskip
\medskip

$removeThing<Dataset>(solidDataset, thing): Dataset \& WithChangeLog$

\bigskip

Rimuove un oggetto contenuto all'interno di un {\tt SolidDataset}.

\bigskip
\medskip

$asUrl(thing, baseUrl): UrlString$

\bigskip

Restituisce l'URL dell'oggetto passato come parametro.

\bigskip
\medskip

$getUrl(thing, property): UrlString | null$

\bigskip

Permette di restituire l'oggetto di uno {\tt statement} contenuto all'interno di un oggetto. È necessario passare a tale funzione l'oggetto e il predicato di tale {\tt risorsa}, cioè il soggetto e il predicato dello {\tt statement} {\tt RDF}. Restituisce il valore {\tt null} se tale {\tt risorsa} non esiste o se il tipo della {\tt risorsa} non è URL. Se esistono più {\tt risorse} di tipo URL corrispondenti a tale predicato, la funzione restituisce un URL fra quelli presenti.

\bigskip
\medskip

$getUrlAll(thing, property): UrlString[]$

\bigskip

Restituisce tutti i valori URL relativi alla proprietà di una {\tt risorsa}.

\bigskip

Esistono inoltre funzioni simili per poter ottenere differenti tipologie di dati, come ad esempio: {\tt string, boolean, decimal, integer} e altre ancora.

\medskip
\bigskip 

$addUrl<T>(thing, property, value): T$

\bigskip

Crea un nuovo oggetto con un URL aggiunto come proprietà; eventuali altri valori relativi a tale proprietà non vengono modificati.

\bigskip

Esistono funzioni simili per poter aggiungere proprietà relative ad altre tipologie di dati, come per esempio: {\tt string, boolean, decimal, integer} e altre.

\medskip
\bigskip

$setUrl<T>(thing, property, value): T$

\bigskip

Crea un nuovo oggetto con gli esistenti valori relativi ad una proprietà rimpiazzati con l'URL passato come parametro.

\bigskip

Esistono funzioni simili per poter sostituire proprietà relative ad altre tipologie di dati, come ad esempio: {\tt string, boolean, decimal, integer} e altre.

\bigskip
\medskip

Per accedere ad alcuni dati relativi alle informazioni sul profilo dell'utente sono stati utilizzati anche i componenti della libreria {\tt Inrupt} {\tt solid-ui-react} $<CombinedDataProvider>$ e $<Text>$.

\bigskip

\section{Modalità di autenticazione con il Solid Identity Provider}

\medskip

Come già accennato riguardo alle specifiche relative all'applicazione {\tt my-solid-blog}, l'applicazione in questione deve poter permettere ai propri utenti di autenticarsi con il proprio {\tt Solid Indentity Provider} per modificare il proprio blog. Per implementare questa funzionalità è stata utilizzata la libreria {\tt Inrupt} {\tt solid-ui-react}, la quale mette a disposizione componenti {\tt React} per autenticarsi e per leggere/scrivere dati.

\bigskip

Vengono ora elencati i componenti appartenenti a questa libreria usati per implementare tale funzionalità:

\bigskip
\medskip

$<SessionProvider>$

\bigskip

Tale componente è usato per avvolgere la propria applicazione; è necessario passare al compomente una {\tt sessionId} universalmente unica. La {\tt sessionId} utilizzata nel caso dell'applicazione {\tt my-solid-blog} è {\tt my-solid-blog}. Tale componente permette di utilizzare all'interno del progetto il componente $<LoginButton>$.

\bigskip
\medskip

$<LoginButton>$

\bigskip

Permette di autenticarsi con il proprio {\tt Solid Identity Provider}, avviando quindi il procedimento di autenticazione descritto all'interno del capitolo 2. È necessario passare a tale componente le seguenti informazioni:

\begin{itemize}
	\item {\tt oidcIssuer}: in questo campo va indicato l'URL del {\tt Solid Identity Provider} con il quale ci si vuole autenticare, come per esempio {\tt https://inrupt.net/}.
	\item {\tt redirectUrl}: indica l'URL sul quale l'utente viene reindirizzato a seguito del processo di autenticazione. Poiché l'applicazione {\tt my-solid-blog} è di tipo single-page e utilizza la porta 3000, il {\tt redirectUrl} usato è pari a {\tt http://localhost:3000/}.
	\item {\tt authOptions}: opzionale, serve ad indicare informazioni aggiuntive relative alla sessione di autenticazione. {\tt my-solid-blog} utilizza tale attributo per indicare il nome dell'applicazione che si sta autenticando con il server {\tt Solid}, che è semplicemente {\tt my-solid-blog}.
\end{itemize}

\bigskip
\medskip

$useSession()$

\bigskip

Funzione contenuta all'interno della libreria {\tt solid-ui-react}, restituisce dati relativi alla sessione di autenticazione; il codice sotto indicato consente di raccogliere tali dati.

\bigskip

{\tt const { session } = useSession();}

\bigskip

L'oggetto {\tt session} contiene quindi dati relativi alla propria sessione di autenticazione, come ad esempio:

\begin{itemize}
	\item {\tt session.info.webId}: contiene la {\tt webId} dell'utente autenticato.
	\item {\tt session.info.isLoggedIn}: proprietà pari a {\tt true} se l'utente ha completato il processo di autenticazione ed è attualmente loggato con il proprio {\tt Solid Identity Provider}, altrimenti tale proprietà è pari a {\tt false}.
	\item {\tt session.fetch}: metodo usato per interagire con i dati contenuti all'interno del {\tt Pod}, usando le informazioni relative al login dell'utente. Tale funzione può essere passata come metodo alle funzioni {\tt getSolidDatasetAt} e {\tt saveSolidDatasetAt} per ottenere i permessi necessari a completare tali operazioni sui dati all'interno del {\tt Pod}.
\end{itemize}

\clearpage

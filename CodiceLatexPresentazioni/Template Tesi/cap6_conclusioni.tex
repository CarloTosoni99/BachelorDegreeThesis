
\chapter{Conclusioni e sviluppi futuri}

\medskip

{\tt Solid} è una tecnologia che nasce a seguito della presa visione di un grave problema riguardante il web, ovvero quello relativo alla perdita di privacy da parte degli utenti di internet. La tecnologia {\tt Solid} mira a risolvere tale problema in maniera molto innovativa, attraverso un nuovo modo di concepire lo stesso web. I vantaggi, da parte dell'utente, nell'utilizzo di applicazioni decentralizzate risultano essere molteplici: in primo luogo attraverso l'utilizzo di {\tt Solid}, viene superato il problema relativo alla perdita di privacy e al controllo dei dati dell'utente da parte delle grandi multinazionali di internet. Inoltre, rappresentando i dati contenuti all'interno dei {\tt Pod} attraverso un unico linguaggio, ovvero {\tt RDF}, è possibile permettere a qualsiasi tipo di applicazione di collaborare nella creazione di un'unica struttura di dati condivisa, in quanto tutti i dati relativi alle applicazioni decentralizzate vengono salvati utilizzando il medesimo formato. La separazione del piano delle applicazioni da quello dei dati, auspicata da Berners-Lee, comporterebbe inoltre per l'utente la possibilità di poter scegliere quali applicazioni web utilizzare, basandosi unicamente sulla qualità del servizio offerto dall'applicazione stessa; infatti, non avendo quest'ultima il controllo dei dati relativi all'utente, non può in alcun modo reclamare i dati che sono già stati salvati all'interno del {\tt Pod}. Da queste considerazioni si può concludere che è possibile, se non auspicabile, che in futuro {\tt Solid}, o un'altra tecnologia simile a questa, possa diventare di uso comune, rimpiazzando l'attuale modello utilizzato per l'interazione con le applicazioni web, il quale prevede la consegna dei propri dati personali in cambio dell'utilizzo gratuito del servizio offerto.

\bigskip

L'apprendimento della programmazione di applicazioni decentralizzate in accordo con questa tecnologia, non è stato per me banale: per poter imparare a programmare applicazioni {\tt Solid} è stato infatti necessario, prima di tutto, imprare a conoscere alcuni linguaggi di programmazione come {\tt HTML}, {\tt CSS} e {\tt JavaScript}, fondamentali per lo sviluppo di applicazioni web, e le tecnologie {\tt React} e {\tt Node.js}. In secondo luogo è stato, ovviamente, necessario imparare a padroneggiare il linguaggio {\tt RDF}, e ad utilizzare le librerie messe a disposizione da {\tt Inrupt}. Oltre a queste considerazioni, è bene anche precisare che, essendo una tecnologia in via di sviluppo, non è presente ancora una documentazione particolarmente dettagliata che consenta di imparare agevolmente a programmare applicazioni decentralizzate tramite {\tt Solid}. Inoltre attualmente non esiste ancora una grande comunità di sviluppatori dietro ad essa; pertanto spesso risulta difficile trovare un supporto in caso di difficoltà. È bene anche menzionare il fatto che, essendo una tecnologia in pieno sviluppo, alcuni aspetti chiave relativi ad essa, come il {\tt Solid Protocol} o il {\tt Web Access Control} citati nei capitoli precedenti, sono ancora soggetti a frequenti modifiche, non avendo attualmente raggiunto uno stadio definitivo. Ovviamente questo comporta allo stato attuale alcune difficoltà allo sviluppatore nel programmare applicazioni in accordo con questa tecnologia. A titolo di esempio all'interno dell'applicazione {\tt my-solid-blog} non è stato per me possibile andare a creare e modificare le risorse {\tt ACL} relative al blog per stabilire livelli di permesso intermedi ad altri utenti; questa funzionalità non è stata, infatti, implementata a causa dello scarso materiale messo a disposizione dagli sviluppatori per imparare a gestire questa particolare tipologia di risorse. Attualmente infatti non esiste una guida sufficientemente dettagliata per risolvere tale problema; per esempio, all'interno della documentazione relativa alla libreria {\tt solid-client} di {\tt Inrupt}, è possibile leggere le seguenti frasi nella descrizione di ogni funzione finalizzata alla gestione delle risorse {\tt ACL}: "La specifica Web Access Control non è ancora stata finalizzata. Come tale, questa funzione è ancora sperimentale e soggetta a cambiamenti, anche in una non-major release" \cite{inruptdoc}. Le problematiche sopra enunciate riguardo alla gestione delle risorse {\tt ACL} hanno fatto sì che l'applicazione {\tt my-solid-blog} abbia potuto usufruire soltanto delle risorse {\tt ACL} di default, contenute all'interno del {\tt Pod}, non potendo andarne a creare delle nuove.

\bigskip

Nonostante le problematiche sopra elencate, {\tt Solid} rimane una tecnologia dal potenziale enorme, in grado di risolvere uno dei problemi più importanti della società di oggi, ovvero quello relativo alla perdita di privacy e alla diffusione di disinformazione online. {\tt Solid} ha acquisito, a partire dalla sua nascita, uno sviluppo e una popolarità sempre maggiori e probabilmente questo andamento continuerà a proseguire nei prossimi anni. In considerazione di quanto esposto all'interno di questo capitolo e di quelli precedenti, mi sento di poter affermare che  una tecnologia come {\tt Solid} è di vitale importanza per migliorare una delle più grandi invenzioni del secolo: il {\tt World Wide Web}, e che nell'utilizzare tale tecnologia gli utenti internet possono trarne grandi benefici.



\clearpage